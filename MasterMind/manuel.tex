\documentclass{beamer}

\usepackage[utf8]{inputenc}
\usepackage{default}
\usepackage[french]{babel}

\usetheme{Warsaw}

\title{Jeu de Mastermind}
\subtitle{Manuel de l'utilisateur}
\author{MEKHILEF Wissame \& RETY Martin}
\institute{Université d'Orléans}
\date{\today}

\begin{document}

  \begin{frame}
   \titlepage
  \end{frame}

  \begin{frame}
   \frametitle{Prérequis}
   \begin{itemize}
    \item L'utilitaire Make
    \item Le package ocaml
   \end{itemize}
   \end{frame}
   
  \begin{frame}
    \frametitle{Premiere fois}
    \begin{itemize}
     \item Se placer à la racine du dossier
     \item Executer en ligne de commande  : make compile
     \item Toujours en ligne de commande  : make run
    \end{itemize}
    Pour relancer une partie il suffit d'executer make run.
  \end{frame}

  \begin{frame}
   \frametitle{Jouer à une partie}
   Une fois le jeu lancer les différentes versions sont accessible par le biais d'un choix à faire au début de la partie:
   \begin{itemize}
    \item 1 : Partie répondant à la question 1, sans redondance de couleur.
    \item 2 : Partie répondant à la question 2, avec une redondance dans les couleurs.
    \item 3 : Partie répondant à la question 3, avec un nombre de coups limité.
   \end{itemize}
  \end{frame}
  
  \begin{frame}
   \frametitle{Déroulement du jeu}
   \begin{itemize}
    \item Composez votre code de 5 couleurs
    \item Renseignez la machine à chacun de ses essaies
    \begin{itemize}
     \item D'abord le nombre de couleurs bien placées
     \item Puis le nombre de couleurs mal placées
    \end{itemize}
   \end{itemize}
  \end{frame}


\end{document}
